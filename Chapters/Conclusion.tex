
\chapter{Conclusion} \label{chap:conclusion}

\section{Discussion}

The goal of this thesis was to create a Constraint Satisfaction Problem Solver from scratch. Within this thesis we developed Oxiflex, a CSP solver from scratch written in Rust. The solver supports the CSP modeling language MiniZinc through implementing a subset of FlatZinc builtin constraints.

We started by discussing what CSPs exactly are by writing them down formally and in the MiniZinc language. Further especially looked at the 8-Queens problem as an easy to understand CSP. Next we discussed possible techniques to solve CSPs using a method called backtracking. We started by using a naive form of the algorithm and improved it further. The first improvement discussed was variable ordering with an fail early approach. That means that we choose variables to assign first, that have the most constraints. We then improved on the backtracking algorithm by adding inference. Starting with forward checking, introducing arc consistency and finally discussing AC-1 and AC-3, arc consistency enforcing algorithms.

Furthermore we gave insights how Oxiflex works and what datastructures are used to solve CSPs. We discussed how the improvements like forward checking and arc consistency work within Oxiflex and what structures are in place to support them.

Finally we measured how each improvement effected both time and number of iterations needed for solving the N-Queens and the Slow Convergence Problems. It is great to see the tradeoff between search and inference though. As we could see in~\cref{fig:slow:sidebyside}. Altough it took longer to solve, inference did reduce the number of iterations significantly. It is interesting to observe the effect that variable ordering has for the slow convergence problem. In fact it made it even possible to solve the problem at all. It can be useful to measure other things than time (like iterations) to gather insights like these.

Therefore the main takeaway for this thesis is that datastructures matter. Altough an algorithm performs better in theory, the right datastructures have to be used to make it really go faster. Using HashMaps to hold all the data migth not be the best approach if performance is the main criteria of a program. This also underlines that just using a fast programming language is not sufficient to make a program go fast.

